% !TeX program = xelatex
% !TeX TXS-program:compile = txs:///xelatex/[--shell-escape]
%%%%%%%%%%%%%%%%%%%%%%%%%%%%%%%%%%%%%%%%%%%%%%%%%%%%%%%%%%%%%%%%%%%%%%%%
% Plantilla TFG/TFM
% Escuela Politécnica Superior de la Universidad de Alicante
% Realizado por: Jose Manuel Requena Plens
% Contacto: info@jmrplens.com / Telegram:@jmrplens
%%%%%%%%%%%%%%%%%%%%%%%%%%%%%%%%%%%%%%%%%%%%%%%%%%%%%%%%%%%%%%%%%%%%%%%%

% Elige si deseas optimizar la ejecución del proyecto almacenando las figuras generadas con TikZ y PGF en una carpeta (archivos/figuras-procesadas).
% 1 - Si, 2 - No
\def\OptimizaTikZ{1}

\input{include/configuracioninicial}

\newcommand{\titulo}{Predicting future body shape before weight loss}
\newcommand{\subtitulo}{Subtítulo del proyecto}

\newcommand{\miNombre}{Pablo Ramón Guevara}
\newcommand{\miEmail}{prg54@alu.ua.es}

\newcommand{\miTutor}{Jorge Azorín López}
\newcommand{\miTutorB}{Andrés Fuster Guilló}
\newcommand{\departamentoTutor}{Department of Computer Science and Technology}
\newcommand{\departamentoTutorB}{Department of Computer Science and Technology}

\newcommand{\miFacultad}{Escuela Politécnica Superior}
\newcommand{\miFacultadCorto}{EPS UA}
\newcommand{\miUniversidad}{\protect{Universidad de Alicante}}
\newcommand{\miUbicacion}{Alicante}

\def\IDtitulo{4}

\input{include/configuraciontitulacion}

% Información añadida a las propiedades del archivo PDF.
\hypersetup{
    pdfauthor = {\miNombre~(\miEmail)},
    pdftitle = {\titulo},
}

%%
% Archivo de acrónimos
%%
\makeglossaries % Genera la base de datos de acrónimos
%%%%%%%%%%%%%%%%%%%%%%%%%%%%%%%%%%%%%%%%%%%%%%%%%%%%%%%%%%%%%%%%%%%%%%%%
% Plantilla TFG/TFM
% Escuela Politécnica Superior de la Universidad de Alicante
% Realizado por: Jose Manuel Requena Plens
% Contacto: info@jmrplens.com / Telegram:@jmrplens
%%%%%%%%%%%%%%%%%%%%%%%%%%%%%%%%%%%%%%%%%%%%%%%%%%%%%%%%%%%%%%%%%%%%%%%%

% Lista de acrónimos (se ordenan por orden alfabético automáticamente)

% La forma de definir un acrónimo es la siguiente:
% \newacronym{id}{siglas}{descripción}
% Donde:
% 	'id' es como vas a llamarlo desde el documento.
%	'siglas' son las siglas del acrónimo.
%	'descripción' es el texto que representan las siglas.
%
% Para usarlo en el documento tienes 4 formas:
% \gls{id} - Añade el acrónimo en su forma larga y con las siglas si es la primera vez que se utiliza, el resto de veces solo añade las siglas. (No utilices este en títulos de capítulos o secciones).
% \glsentryshort{id} - Añade solo las siglas de la id
% \glsentrylong{id} - Añade solo la descripción de la id
% \glsentryfull{id} - Añade tanto  la descripción como las siglas

\newacronym{sota}{SotA}{State of the Art}
\newacronym{nerf}{NeRF}{Neural Radiance Fields}
\newacronym{smpl}{SMPL}{Skinned Multi-Person Linear Model}
\newacronym{rnn}{RNN}{Recurrent Neural Network}
\newacronym{cnn}{CNN}{Convolutional Neural Network}
\newacronym{lstm}{LSTM}{Long Short-Term Memory}
\newacronym{gru}{GRU}{Gated Recurrent Unit}
\newacronym{iwann}{IWANN}{International Work-Conference on Artificial Neural Networks}
\newacronym{pandas}{pandas}{Python Data Analysis Library}
\newacronym{bps}{BPS}{Basis Point Set}
\newacronym{mse}{MSE}{Mean Squared Error}
\newacronym{mae}{MAE}{Mean Average Error}
\newacronym{gpu}{GPU}{Graphics Processing Unit}
\newacronym{gan}{GAN}{Generative Adversarial Network}
\newacronym{vae}{VAE}{Variational Autoencoder}
\newacronym{sgd}{SGD}{Stochastic Gradient Descent}
\newacronym{rmsprop}{RMSProp}{Root Mean Square Propagation}
\newacronym{adam}{Adam}{Adaptive Moment Estimation}
\newacronym{adamw}{AdamW}{Adaptive Moment Estimation with Weight Decay}
\newacronym{vr}{VR}{Virtual Reality}
\newacronym{bmi}{BMI}{Body Mass Index} % Archivo que contiene los acrónimos

%%%%%%%%%%%%%%%%%%%%%%%% 
% INICIO DEL DOCUMENTO
% A partir de aquí debes empezar a realizar tu TFG/TFM
%%%%%%%%%%%%%%%%%%%%%%%%
\begin{document}

% Números romanos hasta el mainmatter.
\frontmatter

% PORTADA
\input{include/portada/portada_color} % Portada Color
\input{include/portada/portada_bn} % Portada B/N

%%%%% PREAMBULO
%%%%%%%%%%%%%%%%%%%%%%%%%%%%%%%%%%%%%%%%%%%%%%%%%%%%%%%%%%%%%%%%%%%%%%%%
% Plantilla TFG/TFM
% Escuela Politécnica Superior de la Universidad de Alicante
% Realizado por: Jose Manuel Requena Plens
% Contacto: info@jmrplens.com / Telegram:@jmrplens
%%%%%%%%%%%%%%%%%%%%%%%%%%%%%%%%%%%%%%%%%%%%%%%%%%%%%%%%%%%%%%%%%%%%%%%%

\chapter*{Preamble}\label{chap:preamble}
\thispagestyle{empty}
% Poner aquí un texto breve que debe incluir entre otras:
% \begin{quote}
%     ``las razones que han llevado a la realización del estudio, el tema, la finalidad y el alcance y también los agradecimientos por las ayudas, por ejemplo apoyo económico (becas y subvenciones) y las consultas y discusiones con los tutores y colegas de trabajo.''
% \end{quote}

% \cleardoublepage %salta a nueva página impar
% \chapter*{Special thanks\footnote{Por si alguien tiene curiosidad, este ``simpático'' agradecimiento está tomado de la ``Tesis de Lydia Chalmers'' basada en el universo del programa de televisión Buffy, la Cazadora de Vampiros.http://www.buffy-cazavampiros.com/Spiketesis/tesis.inicio.htm}
%  }

\thispagestyle{empty}
\vspace{1cm}

Immense gratitude is due to Nahuel, whose unwavering support and guidance
served as the backbone throughout this work.

Further, a particular word of thanks is reserved for my brother Carlos. A
fountain of insight he certainly is, though extracting his wisdom sometimes
felt akin to a light-hearted game of hide and seek. In spite of the gentle
chase required, the value of his guidance cannot be overstated. Here's hoping I
can offer him the same level of invaluable, if slightly elusive, support in the
future.

Finally, my appreciation extends to an exceptional group of individuals who,
while not experts in the field, brought their unique perspectives to review and
refine this project. Their keen eyes and fresh viewpoints were instrumental in
helping me identify and rectify errors that may have otherwise gone unnoticed.

%\cleardoublepage%salta a nueva página impar
% Aquí va la dedicatoria si la hubiese. Si no, comentar la(s) linea(s) siguientes
% \chapter*{}
% \setlength{\leftmargin}{0.5\textwidth}
% \setlength{\parsep}{0cm}
% \addtolength{\topsep}{0.5cm}
% \begin{flushright}
%     \small\em{
%         % A mi esposa Marganit, y a mis hijos Ella Rose y Daniel Adams,\\
%         % sin los cuales habría podido acabar este libro dos años antes \footnote{Dedicatoria de Joseph J. Roman en "An Introduction to Algebraic Topology"}
%     }
% \end{flushright}

% \cleardoublepage %salta a nueva página impar
% Aquí va la cita célebre si la hubiese. Si no, comentar la(s) linea(s) siguientes
% \chapter*{}
% \setlength{\leftmargin}{0.5\textwidth}
% \setlength{\parsep}{0cm}
% \addtolength{\topsep}{0.5cm}
% \begin{flushright}
%     \small\em{
%         Si consigo ver más lejos\\
%         es porque he conseguido auparme\\
%         a hombros de gigantes
%     }
% \end{flushright}
% \begin{flushright}
%     \small{
%         Isaac Newton.
%     }
% \end{flushright}
\cleardoublepage%salta a nueva página impar

% Incluye después del archivo anterior el indice y lista de figuras, tablas y códigos.
\tableofcontents	% Índice
\listoffigures		% Índice de figuras
\listoftables		% Índice de tablas
\lstlistoflistings	% Índice de códigos

% Inicia la numeración habitual.
\mainmatter
%%%%
% CONTENIDO. CAPÍTULOS DEL TRABAJO - Añade o elimina según tus necesidades
%%%%
%%%%%%%%%%%%%%%%%%%%%%%%%%%%%%%%%%%%%%%%%%%%%%%%%%%%%%%%%%%%%%%%%%%%%%%%
% Plantilla TFG/TFM
% Escuela Politécnica Superior de la Universidad de Alicante
% Realizado por: Jose Manuel Requena Plens
% Contacto: info@jmrplens.com / Telegram:@jmrplens
%%%%%%%%%%%%%%%%%%%%%%%%%%%%%%%%%%%%%%%%%%%%%%%%%%%%%%%%%%%%%%%%%%%%%%%%

\chapter{Introduction}

This chapter will introduce the reader to the context of this project. It will
begin by providing a brief overview of the problem at hand, followed by a
description of the motivation behind this work. Finally, it will conclude with
a summary of the contents of the rest of the document.

\section{Motivation and context}

Obesity represents a pressing global public health concern, affecting a
significant segment of the population. \todo{why obesity is bad}.

Tech4Diet is an investigation project that aims to study how obesity treatments
affect morfological changes in the human body. \todo{more about tech4diet}

In a previous initiative, \todo{link} Tech4Diet developed a system enabling
patients undergoing weight loss treatment to visualize 3D scans of their bodies
throughout their weight loss journey~\cite{Azorin-Lopez2020}.

During the treatment, the patient's body is captured using an RGBD camera, the
Intel RealSense D435. The 3D model of the human body is shown during the
different sessions using a virtual reality headset. This innovative approach
aimed to boost motivation and increase treatment plan adherence.

\begin{figure}
	\centering
	\includegraphics[width=75pt]{files/patient_8/8_2}
	\includegraphics[width=75pt]{files/patient_8/8_3}
	\includegraphics[width=75pt]{files/patient_8/8_4}
	\includegraphics[width=75pt]{files/patient_8/8_5}
	\includegraphics[width=75pt]{files/patient_8/8_6}
	\caption[Reconstructed 3D body of a patient's scans]{3D model reconstruction of a patient's body at different stages of a weight loss
		treatment. There is around a month between each scan, and a total
		weight loss of 3.8 kg.}
\end{figure}

Besides body scans, the study also collected other medical data, including
variables such as weight, localized fat and muscle mass, activity levels and
other psychological factors.

Subsequently, we wondered if it would be feasible to utilize the datasets
acquired in this prior study to formulate a predictive model. This model would
project anticipated changes in a person's body undergoing weight loss treatment
before the treatment concludes, further bolstering adherence to the treatment
regimen.

The present work explores the development of such a model. This includes
analyzing data from the earlier study, reviewing existing techniques in human
body model representation, encoding patient data using the chosen
representation, devising a neural network architecture for predicting patient
body changes, training and evaluating the model and finally, generating 3D
meshes of the predicted body changes.

We will expand on the details of this process in the following chapters.
Chapter \nameref{chap:data} will delve into the data collected during the
previous study, and how we processed it to prepare it for use in our model, as
well as how we encoded the human body scans. Afterwards, chapter
\nameref{chap:nn} will go over the neural network architecture we devised for
this project, as well as the training process and the results obtained.
Finally, chapter \nameref{chap:results} will discuss the results of our model,
and chapter \nameref{chap:conclusion} will conclude the document with a summary
of the work done and possible future lines of research.

\section{Background}

As previously mentioned, this section will provide a brief overview of the
\gls{sota} in the field of human body representation and generative neural
network architectures. We were able to use what we learned from the research on
3D human body models to write and submit a paper to the \gls{iwann} 2023
conference. \todo{link}

\subsection{3D human body representation}

The field of 3D human body recovery has seen a significant advancement with the
development of parametric models. These methods use a set of parameters to
represent body shape and pose and are widely used for reconstructing 3D human
body. These methods have different features, with some focusing on body
deformations, others on shape and pose optimization, and others on the
separation of body shape into identity-specific and pose-dependent components,
among other things. The advancements in the field have led to improved accuracy
and stability in representing human body shapes and poses. On the other hand,
in recent years various generative methods have been developed to generate 3D
models of the human body. Variational Autoencoder (VAEs) and Generative
Adversarial Networks (GANs) are two commonly used types of neural networks for
this purpose. These methods can generate 3D human body models by learning the
distribution of the data. There are many areas that can make use of these
models. Some of the most significant applications include:

\begin{itemize}
	\item Medicine: Human body models are valuable in the study of
	      anatomy~\cite{https://doi.org/10.1002/ase.1718} and for patient monitoring.
	\item Film industry: Human body models can be used to capture motion data and render
	      high-quality CGI humans.
	\item Video game industry: Human body models can be used to create realistic
	      animations and interactions between characters\cite{Starke2021}.
	\item Extended reality: Human body models can be used to capture user input in
	      virtual reality as well as rendering realistic characters.
	\item Clothing: These models can be used for fitting virtual
	      clothes\cite{apeagyei2010application} and creating realistic images of clothing
	      products.
\end{itemize}

A straightforward approach to categorize human body representations is to
classify based on the required input type and the generated output.

Regarding input, representations fall into three categories:

\begin{itemize}
	\item 2D input: These representations utilize 2D images or videos as input.
	      Some models may also process images from varying angles. The flexibility
	      of these models is beneficial as they do not necessitate specific hardware
	      to capture the input data.
	\item 3D input: Typically, these models require 3D point clouds as input data.
	\item Parametric models: These models demand a set of parameters describing the body.
	      Some models categorize these parameters into body shape and body pose. This
	      form of representation is highly intriguing for machine learning applications
	      due to the significant reduction in input data dimensionality. This factor
	      permits the training of a neural network with fewer samples. However, it
	      necessitates a model capable of generating parameters from the input data.
\end{itemize}

Some of the most common 3D output types are:

\begin{itemize}
	\item 3D meshes: These models create a 3D mesh of the object.
	\item 3D voxel: These models produce a 3D voxel grid of the object.
	      However, this approach is not widespread as it is generally not
	      beneficial for most applications.
	\item \gls{nerf}: This novel 3D representation directly renders the object
	      from a specific viewpoint. While this allows for the generation of
	      highly realistic images, it proves less beneficial for applications
	      requiring a true 3D representation of the object.
\end{itemize}

\subsubsection{Generation}

When it comes to generating 3D human models, we distinguish two main
approaches. The first approach is to use a general purpose generator system and
guide it to generate human models. The second approach is to use a generator
that has been specifically designed to generate human models from the start.

\paragraph{Human specific}

SiCloPe \cite{SiCloPe} models clothed human bodies using deep generative
models. It can reconstruct a complete and textured 3D model of a person wearing
clothes from a single input picture. It uses a silhouette-based representation
that combines 2D silhouettes and 3D joints of a body pose to describe the
complex shape variations of clothed people. It synthesizes consistent
silhouettes and feeds them into a deep visual hull algorithm for 3D shape
prediction and uses a conditional generative adversarial network to infer the
texture of the subject's back view.

PIFu \cite{PIFu} is a highly effective implicit representation that locally
aligns pixels of 2D images with their corresponding 3D object. The method can
infer 3D surface and texture from a single image or multiple input images. It
can handle intricate shapes and their variations and deformations, and can
produce high-resolution surfaces including largely unseen regions such as the
back of a person. It extends naturally to arbitrary number of views and is
memory efficient, spatially aligned with the input image and can handle
arbitrary topology. PIFuHD \cite{PIFuHD} builds on top of PIFu with an
additional module and applies it to the task of human digitalization.

Tex2Shape \cite{Tex2Shape} is a simple method to infer detailed full human body
shape from a single photograph. It turns shape regression into an aligned
image-to-image translation problem and estimates detailed normal and vector
displacement maps from partial texture maps of the visible region. The results
feature details even on parts that are occluded in the input image and the
model generalizes well to real-world photographs.

HumanMeshNet \cite{HumanMeshNet} regresses a template mesh's vertices and
receives regularization from 3D skeletal locations in a multi-branch,
multi-task framework. It focuses on implicitly learning the mesh representation
and is a novel model for 3D human body reconstruction from a monocular image.

DeepHuman \cite{DeepHuman} is image-guided 3D human reconstruction network that
leverages a dense semantic representation and fuses different scales of image
features into the 3D space. The visible surface details are refined through a
normal refinement network. The method outperforms state-of-the-art approaches
in 3D human model estimation from a single image.

HumanGen \cite{humangen} is a 3D human generation scheme with detailed geometry
and 360° realistic free-view rendering. The scheme marries the 3D human
generation with various priors from the 2D generator and 3D reconstructor of
humans through the design of an "anchor image." The authors adopt a pronged
design to disentangle the generation of geometry and appearance and use an
anchor image to adapt a 3D reconstructor for fine-grained details synthesis and
propose a two-stage generation scheme for geometry and appearance.

HumanNeRF \cite{humannerf} describes a neural representation for high-fidelity
free-view synthesis of dynamic humans. It uses an aggregated pixel-alignment
feature with a pose embedded non-rigid deformation field and raw HumanNeRF can
already produce reasonable rendering on sparse video inputs. The approach is
improved with in-hour scene-specific fine-tuning and appearance blending. The
authors show that this approach is effective in synthesizing photorealistic
free-view humans with sparse camera view inputs.

\paragraph{General}

The CoCosNet \cite{CoCosNet} (and CoCosNet v2 \cite{CoCosNet2}) paper
introduces full-resolution correspondence learning for cross-domain image
translation. It uses a hierarchical strategy that employs the correspondence
from coarse to fine levels and utilizes the ConvGRU module to refine the
current correspondence. The result is a highly efficient and effective approach
for exemplar-based image translation that outperforms state-of-the-art
literature.

\subsubsection{Generative neural networks}

\section{Objectives}\label{objectives}

\begin{itemize}
	\item \textbf{Objective 1} Study the state of the art in human body representation and
	      generation. \subitem Review the literature on human body representation and
	      generation. \subitem Analyze the advantages and disadvantages of the different
	      approaches. \subitem Select the most suitable approach for the project.
	\item \textbf{Objective 2} Analyze and process the data to be used in the project. \subitem
	      Explore the data and its characteristics. \subitem Preprocess the data to be
	      used in the project, identifying and solving any issues that may arise.
	      \subitem Study what data augmentation techniques can be applied to the data.
	\item \textbf{Objective 3} Design and implement a neural network that can understand human
	      bodies and train it to predict shape changes in time. \subitem Study the
	      different neural network architectures that can be used in a time series
	      prediction problem. \subitem Design a neural network architecture that can be
	      used to generate future human body shapes. \subitem Implement the neural
	      network architecture. \subitem Train the neural network. \subitem Evaluate the
	      predictions made by the neural network.
	\item \textbf{Objective 4} Evaluate the results. \subitem Generate predicted human body
	      shapes for a set of input data. \subitem Evaluate the predictions.
\end{itemize}


\chapter{Data analysis and preprocessing}\label{data}

Our data comprises approximately 400 sessions collected from 80 patients. These
sessions were recorded at various time points and are not uniformly spaced.
Each session's data includes a 3D scan of the patient, along with several
measurements such as weight, height, body fat percentage, etc.

\begin{table}[h]
    \centering
    \begin{tabular}{c c l c}
        \toprule
        Type                               & Source                                                                                         & Measurement (unit)                        \\
        \midrule
        \multirow{3}{*}{Anthropometric}    & \multirow{3}{4cm}{flexible measuring tape}                                                     & Wrist (cm) (kg)                           \\
                                           &                                                                                                & Waist (cm)                                \\
                                           &                                                                                                & Hip (cm)                                  \\
        \midrule

        \multirow{6}{*}{Body composition}  & \multirow{6}{4cm}{Tanita\textregistered\ MC 780-P MA and Seca\textregistered\ 213 stadiometer} & Fat per limb and trunk (\%)               \\
                                           &                                                                                                & Muscle per limb and trunk (\%)            \\
                                           &                                                                                                & Total fat and muscle (\%)                 \\
                                           &                                                                                                & Visceral fat area (cm\textsuperscript{2}) \\
                                           &                                                                                                & Weight (kg)                               \\
                                           &                                                                                                & Height (m)                                \\
        \midrule

        \multirow{3}{*}{Other, Lifestyle}  & \multirow{3}{4cm}{Interview}                                                                   & Activity   (score)                        \\
                                           &                                                                                                & Gender                                    \\
                                           &                                                                                                & Age (years)                               \\

        \midrule

        \multirow{3}{*}{Blood (capillary)} & \multirow{3}{4cm}{Accutrend\textregistered\ Plus}                                              & Glucose (mg/dL)                           \\
                                           &                                                                                                & Cholesterol (mg/dL)                       \\
                                           &                                                                                                & Triglycerides (mg/dL)                     \\

        \midrule

        \multirow{2}{*}{Blood pressure}    & \multirow{2}{4cm}{Omron\textregistered\ M3}                                                    & Systolic pressure (mmHg)                  \\
                                           &

                                           & Diastolic pressure (mmHg)                                                                                                                  \\
        \bottomrule

    \end{tabular}
    \caption{Measurements collected from each session.}
\end{table}

\todo[]{${https://rua.ua.es/dspace/bitstream/10045/124160/6/Garcia-dUrso_etal_2022_IEEEAccess.pdf}$}

However, the data requires cleaning before usage. Some sessions lack certain
measurements, and there are numerous outliers within the data. After cleaning,
we utilized about 200 sessions.

\begin{figure}[h]
    \centering
    \includegraphics[width=0.5\textwidth]{files/sessions_per_patient}
    \caption{Number of sessions per patient}
    \label{fig:sessions-per-patient}
\end{figure}

\subsection{Data cleaning}

We developed a data cleaning pipeline using the \gls{pandas} library. This
pipeline fixes some errors in the data, removes outliers and sessions with
missing measurements.

In order to do this data cleaning, we indexed the dataset with a multi-index
containing the patient's id and the number of the session. This allowed us to
iterate over the data in a per-patient basis, and to easily identify patients
with problems in their data.

After plotting and examining the data, we identified that one of the most
common issues was due to the decimal separator. Many measurements were off by a
factor of 10, so we implemented custom rules to solve this issue. We also
implemented rules to locate and remove values that were out of range or that
didn't match to other measurements.

Some of them are:

\begin{itemize}
    \item Variation between measurements of different limbs. For example, the difference
          in muscle or fat percentage between the left and right arm should be small.
    \item The sum of the fat and muscle percentages should be less than 100\%.
    \item Percentage levels of fat and muscle should be less than 100\%. If they are,
          that probably means that the measurement is off by a factor of 10.
\end{itemize}

Even after fixing these issues there were still some outliers. We tested a
system that detected extreme changes between sessions, but this was too prone
to false positives. In the end, we opted to manually visualize the data and
mark these outliers for removal.

\begin{figure}[h]
    \centering
    \begin{subfigure}{\textwidth}
        \centering
        \includegraphics[width=0.8\textwidth]{files/weight_unfiltered}
        \caption{Raw weight}
    \end{subfigure}
    \begin{subfigure}{\textwidth}
        \centering
        \includegraphics[width=0.8\textwidth]{files/weight_filtered}
        \caption{Filtered weight}
    \end{subfigure}
    \caption{Weight measurements before and after filtering}
\end{figure}

\section{Body representation}

In order to represent the body shape we opted to use \gls{smpl}. \gls{smpl}
encodes the body shape and pose using a low-dimensional linear space. The body
shape is encoded using 10 shape parameters ($\beta$), and the pose is encoded
using 72 pose parameters ($\theta$). As our interest lies in body shape, we
employed only the shape parameters.

We extracted \gls{smpl} parameters --- shape ($\beta$) and pose ($\theta$) ---
from the 3D scans using a custom minimization algorithm.

\def\betaVar{3}
\def\imgWidth{0.3\textwidth}
\def\betaWidth{\textwidth}

\begin{figure}[ht!]
    \centering
    \begin{minipage}[b]{\textwidth}
        \centering
        \includegraphics[width=\imgWidth]{files/visualize_betas/beta_0_-\betaVar_m}
        \includegraphics[width=\imgWidth]{files/visualize_betas/baseline_m}
        \includegraphics[width=\imgWidth]{files/visualize_betas/beta_0_\betaVar_m}
        \linebreak
        \includegraphics[width=\imgWidth]{files/visualize_betas/beta_0_-\betaVar_f}
        \includegraphics[width=\imgWidth]{files/visualize_betas/baseline_f}
        \includegraphics[width=\imgWidth]{files/visualize_betas/beta_0_\betaVar_f}
        \caption{$\beta_1 = [-\betaVar, 0, +\betaVar]$}
        \label{fig:beta-1-vis}
    \end{minipage}
\end{figure}

\begin{figure}[ht!]
    \centering

    \begin{minipage}[b]{\textwidth}
        \centering
        \includegraphics[width=\imgWidth]{files/visualize_betas/beta_1_-\betaVar_m}
        \includegraphics[width=\imgWidth]{files/visualize_betas/baseline_m}
        \includegraphics[width=\imgWidth]{files/visualize_betas/beta_1_\betaVar_m}
        \linebreak
        \includegraphics[width=\imgWidth]{files/visualize_betas/beta_1_-\betaVar_f}
        \includegraphics[width=\imgWidth]{files/visualize_betas/baseline_f}
        \includegraphics[width=\imgWidth]{files/visualize_betas/beta_1_\betaVar_f}
        \caption{$\beta_2 = [-\betaVar, 0, +\betaVar]$}
    \end{minipage}
\end{figure}

\begin{figure}[ht!]
    \centering

    \begin{minipage}[b]{\textwidth}
        \centering
        \includegraphics[width=\imgWidth]{files/visualize_betas/beta_2_-\betaVar_m}
        \includegraphics[width=\imgWidth]{files/visualize_betas/baseline_m}
        \includegraphics[width=\imgWidth]{files/visualize_betas/beta_2_\betaVar_m}
        \linebreak
        \includegraphics[width=\imgWidth]{files/visualize_betas/beta_2_-\betaVar_f}
        \includegraphics[width=\imgWidth]{files/visualize_betas/baseline_f}
        \includegraphics[width=\imgWidth]{files/visualize_betas/beta_2_\betaVar_f}
        \caption{$\beta_3 = [-\betaVar, 0, +\betaVar]$}
    \end{minipage}
\end{figure}

\begin{figure}[ht!]
    \centering

    \begin{minipage}[b]{\textwidth}
        \centering
        \includegraphics[width=\imgWidth]{files/visualize_betas/beta_3_-\betaVar_m}
        \includegraphics[width=\imgWidth]{files/visualize_betas/baseline_m}
        \includegraphics[width=\imgWidth]{files/visualize_betas/beta_3_\betaVar_m}
        \linebreak
        \includegraphics[width=\imgWidth]{files/visualize_betas/beta_3_-\betaVar_f}
        \includegraphics[width=\imgWidth]{files/visualize_betas/baseline_f}
        \includegraphics[width=\imgWidth]{files/visualize_betas/beta_3_\betaVar_f}
        \caption{$\beta_4 = [-\betaVar, 0, +\betaVar]$}
    \end{minipage}
\end{figure}

\begin{figure}[ht!]
    \centering

    \begin{minipage}[b]{\textwidth}
        \centering
        \includegraphics[width=\imgWidth]{files/visualize_betas/beta_4_-\betaVar_m}
        \includegraphics[width=\imgWidth]{files/visualize_betas/baseline_m}
        \includegraphics[width=\imgWidth]{files/visualize_betas/beta_4_\betaVar_m}
        \linebreak
        \includegraphics[width=\imgWidth]{files/visualize_betas/beta_4_-\betaVar_f}
        \includegraphics[width=\imgWidth]{files/visualize_betas/baseline_f}
        \includegraphics[width=\imgWidth]{files/visualize_betas/beta_4_\betaVar_f}
        \caption{$\beta_5 = [-\betaVar, 0, +\betaVar]$}
    \end{minipage}
\end{figure}

\begin{figure}[ht!]
    \centering

    \begin{minipage}[b]{\textwidth}
        \centering
        \includegraphics[width=\imgWidth]{files/visualize_betas/beta_5_-\betaVar_m}
        \includegraphics[width=\imgWidth]{files/visualize_betas/baseline_m}
        \includegraphics[width=\imgWidth]{files/visualize_betas/beta_5_\betaVar_m}
        \linebreak
        \includegraphics[width=\imgWidth]{files/visualize_betas/beta_5_-\betaVar_f}
        \includegraphics[width=\imgWidth]{files/visualize_betas/baseline_f}
        \includegraphics[width=\imgWidth]{files/visualize_betas/beta_5_\betaVar_f}
        \caption{$\beta_6 = [-\betaVar, 0, +\betaVar]$}
    \end{minipage}
\end{figure}

\begin{figure}[ht!]
    \centering

    \begin{minipage}[b]{\textwidth}
        \centering
        \includegraphics[width=\imgWidth]{files/visualize_betas/beta_6_-\betaVar_m}
        \includegraphics[width=\imgWidth]{files/visualize_betas/baseline_m}
        \includegraphics[width=\imgWidth]{files/visualize_betas/beta_6_\betaVar_m}
        \linebreak
        \includegraphics[width=\imgWidth]{files/visualize_betas/beta_6_-\betaVar_f}
        \includegraphics[width=\imgWidth]{files/visualize_betas/baseline_f}
        \includegraphics[width=\imgWidth]{files/visualize_betas/beta_6_\betaVar_f}
        \caption{$\beta_7 = [-\betaVar, 0, +\betaVar]$}
    \end{minipage}
\end{figure}

\begin{figure}[ht!]
    \centering

    \begin{minipage}[b]{\textwidth}
        \centering
        \includegraphics[width=\imgWidth]{files/visualize_betas/beta_7_-\betaVar_m}
        \includegraphics[width=\imgWidth]{files/visualize_betas/baseline_m}
        \includegraphics[width=\imgWidth]{files/visualize_betas/beta_7_\betaVar_m}
        \linebreak
        \includegraphics[width=\imgWidth]{files/visualize_betas/beta_7_-\betaVar_f}
        \includegraphics[width=\imgWidth]{files/visualize_betas/baseline_f}
        \includegraphics[width=\imgWidth]{files/visualize_betas/beta_7_\betaVar_f}
        \caption{$\beta_8 = [-\betaVar, 0, +\betaVar]$}
    \end{minipage}
\end{figure}

\begin{figure}[ht!]
    \centering

    \begin{minipage}[b]{\textwidth}
        \centering
        \includegraphics[width=\imgWidth]{files/visualize_betas/beta_8_-\betaVar_m}
        \includegraphics[width=\imgWidth]{files/visualize_betas/baseline_m}
        \includegraphics[width=\imgWidth]{files/visualize_betas/beta_8_\betaVar_m}
        \linebreak
        \includegraphics[width=\imgWidth]{files/visualize_betas/beta_8_-\betaVar_f}
        \includegraphics[width=\imgWidth]{files/visualize_betas/baseline_f}
        \includegraphics[width=\imgWidth]{files/visualize_betas/beta_8_\betaVar_f}
        \caption{$\beta_9 = [-\betaVar, 0, +\betaVar]$}
    \end{minipage}
\end{figure}

\begin{figure}[ht!]
    \centering

    \begin{minipage}[b]{\textwidth}
        \centering
        \includegraphics[width=\imgWidth]{files/visualize_betas/beta_9_-\betaVar_m}
        \includegraphics[width=\imgWidth]{files/visualize_betas/baseline_m}
        \includegraphics[width=\imgWidth]{files/visualize_betas/beta_9_\betaVar_m}
        \linebreak
        \includegraphics[width=\imgWidth]{files/visualize_betas/beta_9_-\betaVar_f}
        \includegraphics[width=\imgWidth]{files/visualize_betas/baseline_f}
        \includegraphics[width=\imgWidth]{files/visualize_betas/beta_9_\betaVar_f}
        \caption{$\beta_{10} = [-\betaVar, 0, +\betaVar]$}
    \end{minipage}
\end{figure}


Figure~\ref{fig:beta-vis} shows the effect of varying the shape parameters. We
have used a scale of 3 to better show their effect, but in practice the values
are much smaller. $\beta_0$ controls the overall height of the body, while
$\beta_1$ has a large correlation with the body mass index.

\begin{table}[h]
    \centering
    \begin{tabular}{c | c c c c c c c c c c}
        \toprule
             & $\beta_1$ & $\beta_2$ & $\beta_3$ & $\beta_4$ & $\beta_5$ & $\beta_6$ & $\beta_7$ & $\beta_8$ & $\beta_9$ & $\beta_{10}$ \\
        \midrule
        mean & 0.88      & -0.73     & 0.34      & 0.01      & 0.06      & 0.06      & 0.11      & 0.02      & 0.01      & 0.11         \\

        std  & 0.97      & 0.78      & 0.26      & 0.21      & 0.12      & 0.13      & 0.08      & 0.03      & 0.03      & 0.08         \\

        min  & -1.42     & -2.57     & -0.64     & -0.65     & -0.23     & -0.25     & -0.14     & -0.07     & -0.08     &
        -0.17                                                                                                                           \\

        25\% & 0.12      & -1.26     & 0.15      & -0.14     & -0.01     & -0.02     & 0.04      & 0.00      & -0.01     & 0.06         \\

        50\% & 0.94      & -0.69     & 0.36      & 0.03      & 0.04      & 0.02      & 0.11      & 0.02      & 0.01      & 0.11         \\

        75\% & 1.66      & -0.20     & 0.52      & 0.17      & 0.14      & 0.14      & 0.16      & 0.04      & 0.04      & 0.17         \\

        max  & 2.90      & 2.96      & 0.97      & 0.45      & 0.39      & 0.47      & 0.36      & 0.13      & 0.16      & 0.30         \\
        \bottomrule
    \end{tabular}
    \caption{Statistics of the shape parameters}
\end{table}
\chapter{Neural network}\label{nn}

\section{Neural network architecture}

Given the nature of the data, we decided to use a neural network to predict the
changes in the body shape. Since the data is temporal, we need to use a neural
network architecture that can handle temporal data.

There are different neural network architectures that work well with temporal
data. Recurrent neural networks are a type of neural networks that feed the
output of the previous step as input to the next step. This allows them to
remember information from previous steps, which is useful for time series.
However, they can `forget' information from the beginning of the sequence,
which is a problem known as vanishing gradients. There are some variations of
recurrent neural networks that try to solve this problem, such as \gls{lstm}
and \gls{gru}.

Transformer networks are a relatively new type of neural network that has been
used with great success in natural language processing. They are based on
attention mechanisms, which allow them to focus on specific parts of the input
sequence. This makes them very useful for time series, since they can focus on
the most relevant parts of the sequence. The big disadvantage of transformer
networks for this application is that they require large amounts of training
data\todo[]{citation needed}, which is not available in this case.

We ended up using an neural network architecture that uses an \gls{lstm}.

\section{Training}

\subsection{Variability in the dates of the sessions}

The dataset's sessions are not uniformly spaced in time, varying from a few
days to several months apart. To mitigate this issue, we:

Calculated a variable representing the number of days until the next session.

Modified the neural network to predict the daily change in variables, instead
of predicting the variables for the following session.

We implemented a residual connection to the neural network, enabling it to
predict the change in variables from one session to the next. Then, we
multiplied this change by the number of days until the next session and added
it to the previous session to calculate the predicted values for the next
session. This approach assumes a linear change between sessions—an
approximation suited to our purpose. We augmented the data by randomly removing
intermediate sessions and recalculating the number of days until the next
session.

\section{Neural network architecture}

We used the PyTorch library to implement our custom neural network
architecture.

\begin{figure}
    \centering
    \includegraphics[width=6cm]{files/nn_diagram}
    \caption{Diagram of the neural network architecture}
\end{figure}

This architecture can handle temporal data of varying lengths.

\section{Training}

The neural network inputs a tensor of shape (batch size, max sequence length,
number of features), a tensor of shape (batch size, 1) containing the length of
the current sequence, and a scalar representing the number of days until the
next session. It returns a tensor of shape (batch size, max sequence length,
number of features) containing the predicted values for the next session.

\begin{itemize}
    \item The neural network takes as input:
          \begin{itemize}
              \item A tensor of shape (batch size, max sequence length, number of features).
              \item A tensor of shape (batch size, 1) containing the length of the current
                    sequence.
              \item A scalar representing the number of days until the next session.
          \end{itemize}
    \item The neural network returns:
          \begin{itemize}
              \item A tensor of shape (batch size, max sequence length, number of features) that
                    contains the predicted values for the next session.
          \end{itemize}
\end{itemize}

For features, we concatenate the \gls{smpl} parameters shape ($\beta$) with the
patient's height, weight, and age. We also experimented with using body fat
percentage and muscle mass percentage, but the results were comparable. We used
the AdamW optimizer with a variable learning rate and weight decay, and MSE
loss.

\section{Evaluation}

The mean absolute error (MAE) of the predicted betas served as our evaluation
metric.

\section{Results}

\begin{figure}[h]
    \centering
    \includegraphics[width=\textwidth]{files/predicted_betas}
    \caption{Change of the shape parameters for a given patient and the neural
        network prediction after 30 days}
\end{figure}
%%%%%%%%%%%%%%%%%%%%%%%%%%%%%%%%%%%%%%%%%%%%%%%%%%%%%%%%%%%%%%%%%%%%%%%%
% Plantilla TFG/TFM
% Escuela Politécnica Superior de la Universidad de Alicante
% Realizado por: Jose Manuel Requena Plens
% Contacto: info@jmrplens.com / Telegram:@jmrplens
%%%%%%%%%%%%%%%%%%%%%%%%%%%%%%%%%%%%%%%%%%%%%%%%%%%%%%%%%%%%%%%%%%%%%%%%

\chapter{Human Body Prediction Results}\label{chap:results}

In this chapter we will go through the results obtained from the experiments
performed in the previous chapter. We will start by evaluating the predictive
performance of our model, followed by a discussion of the results of generated
bodies and their implications.

\section{Evaluation}

\begin{figure}
	\centering
	\includegraphics[width=0.3\textwidth]{files/predictions_beta/beta_1.png}
	\includegraphics[width=0.3\textwidth]{files/predictions_beta/beta_2.png}
	\includegraphics[width=0.3\textwidth]{files/predictions_beta/beta_3.png}
	\includegraphics[width=0.3\textwidth]{files/predictions_beta/beta_4.png}
	\includegraphics[width=0.3\textwidth]{files/predictions_beta/beta_5.png}
	\includegraphics[width=0.3\textwidth]{files/predictions_beta/beta_6.png}
	\includegraphics[width=0.3\textwidth]{files/predictions_beta/beta_7.png}
	\includegraphics[width=0.3\textwidth]{files/predictions_beta/beta_8.png}
	\includegraphics[width=0.3\textwidth]{files/predictions_beta/beta_9.png}
	\includegraphics[width=0.3\textwidth]{files/predictions_beta/beta_10.png}
	\caption[Predictions vs ground truth]{Predictions (y-axis) vs ground truth (x-axis) for each $\beta$ parameter.}
	\label{fig:scatter}
\end{figure}

\subsection{Quantitative results}

The mean \gls{mae} of the model in the test set when predicting the $\beta$
parameter is \textbf{0.064}. Figure \ref{fig:scatter} shows the predictions of
the model vs the ground truth, and Figure \ref{fig:fold-mae} shows the
\gls{mae} for each $\beta$ per fold. We can see that the model is able to
predict the correct value in most cases, but it has some problems with
outliers.

\begin{figure}[H]
	\centering
	\includegraphics[width=300pt]{files/mae_folds.png}
	\caption{MAE for each $\beta$ per fold.}
	\label{fig:fold-mae}
\end{figure}

\begin{figure}[H]
	\centering
	\includegraphics[width=\textwidth]{files/compounded.png}
	\caption{Compounded predictions for 180 days.}
	\label{fig:compounded}
\end{figure}

However, Figure \ref{fig:compounded} shows that the model's predictions begin
to diverge once we try to feed the predictions back into the model. This makes
it unsuitable for long-term predictions without some form of regularization.

One possible explanation is that we are trying to predict more into the future
than the timescale of our scans. We believe that obtaining more scans would
help the model to learn the long-term dynamics of the patient.

\subsection{Qualitative results}

Although it is hard to evaluate the quality of the generated bodies, we can see
that the model is able to generate bodies that look like the patient's body.
Figure {\ref{fig:pred-1}} shows that the predicted bodies after 30 days look
plausible.

As we saw in the previous chapter, Figure \ref{fig:predicted-betas} shows that
the value for $\beta_2$, inversely correlated with \gls{bmi}, tends to
increase, so the predicted bodies tend to be thinner.

In spite of this, the model is unstable if we try to use the predictions as
input for the next prediction. Figure \ref{fig:compounded} shows that the
predicted values for the $\beta$ parameter diverge. However, if we take a look
at the predicted bodies, in Figure {\ref{fig:pred-2}} we can see that the
predicted bodies still look plausible.

\begin{figure}[h]
	\centering
	\begin{subfigure}{\textwidth}
		\centering
		\includegraphics[width=60pt]{files/patients/9_2}
		\includegraphics[width=60pt]{files/patients/9_3}
		\includegraphics[width=60pt]{files/patients/9_4}
		\includegraphics[width=60pt]{files/patients/9_5}
		\includegraphics[width=60pt]{files/patients/9_6}
		\hspace{10pt}
		\includegraphics[width=60pt]{files/patients/9_predicted}
	\end{subfigure}
	\linebreak
	\begin{subfigure}{\textwidth}
		\centering
		\includegraphics[width=60pt]{files/patients/128_1}
		\includegraphics[width=60pt]{files/patients/128_2}
		\includegraphics[width=60pt]{files/patients/128_3}
		\includegraphics[width=60pt]{files/patients/128_4}
		\includegraphics[width=60pt]{files/patients/128_5}
		\hspace{10pt}
		\includegraphics[width=60pt]{files/patients/128_predicted}
	\end{subfigure}
	\caption{Predicted bodies for two patients.}
	\label{fig:pred-1}
\end{figure}

\begin{figure}[h]
	\centering
	\begin{subfigure}{0.3\textwidth}
		\centering
		\includegraphics[width=120pt]{files/patients/2_predicted_2.png}
		\caption{93.8 kg}
	\end{subfigure}
	\begin{subfigure}{0.3\textwidth}
		\centering
		\includegraphics[width=120pt]{files/patients/2_predicted_3.png}
		\caption{93.1 kg}
	\end{subfigure}
	\begin{subfigure}{0.3\textwidth}
		\centering
		\includegraphics[width=120pt]{files/patients/2_predicted_4.png}
		\caption{92.4 kg}
	\end{subfigure}
	\begin{subfigure}{0.3\textwidth}
		\centering
		\includegraphics[width=120pt]{files/patients/2_predicted_5.png}
		\caption{91.6 kg}
	\end{subfigure}
	\begin{subfigure}{0.3\textwidth}
		\centering
		\includegraphics[width=120pt]{files/patients/2_predicted_6.png}
		\caption{90.8 kg}
	\end{subfigure}
	\begin{subfigure}{0.3\textwidth}
		\centering
		\includegraphics[width=120pt]{files/patients/2_predicted_7.png}
		\caption{90.0 kg}
	\end{subfigure}
	\caption[Compounded generations]{Compounded generations for a patient after 30, 60, 90, 120, 150, and 180 days.}
	\label{fig:pred-2}
\end{figure}

\section{Discussion}

\begin{figure}[h]
	\centering
	\includegraphics[width=0.3\textwidth]{files/beta_var/beta_1_var.png}
	\includegraphics[width=0.3\textwidth]{files/beta_var/beta_2_var.png}
	\includegraphics[width=0.3\textwidth]{files/beta_var/beta_3_var.png}
	\includegraphics[width=0.3\textwidth]{files/beta_var/beta_3_var.png}
	\includegraphics[width=0.3\textwidth]{files/beta_var/beta_5_var.png}
	\includegraphics[width=0.3\textwidth]{files/beta_var/beta_6_var.png}
	\includegraphics[width=0.3\textwidth]{files/beta_var/beta_7_var.png}
	\includegraphics[width=0.3\textwidth]{files/beta_var/beta_8_var.png}
	\includegraphics[width=0.3\textwidth]{files/beta_var/beta_9_var.png}
	\includegraphics[width=0.3\textwidth]{files/beta_var/beta_10_var.png}

	\caption{Variation of $\beta$ per patient.}
	\label{fig:beta-var}
\end{figure}

One of the main problems of the model is that it has a lot of variance in
$\beta_1$. This parameter controls the overall height of the person (Figure
\ref{fig:beta-1-vis}). This has the effect of making the generated human bodies
vary in height, which is not desirable.

This is probably due to the fact that the training data has a lot of variance
in height, which is probably due to scanning error or our method of extracting
the \gls{smpl} parameters from the scans. Figure \ref{fig:beta-var} shows the
variation of $\beta_1$ in the dataset per patient.

To mitigate this problem, several solutions can be explored. One potential
solution is the improvement of data preprocessing, specifically in the
extraction of the SMPL parameters from the scans. By refining this process, the
quality of the training data can be significantly enhanced, reducing the
variance in the $\beta_1$ parameter and leading to more accurate predictions.

Alternatively, we could consider incorporating a form of regularization into
our model specifically targeted at controlling the variation in the height
parameter. By including a penalty term in our loss function that encourages
consistency in the height parameter, we can influence the model to maintain
more stable height predictions across sequences.

Finally, it is also possible to investigate the use of post-processing
techniques. For instance, once the model makes a prediction, we can adjust the
$\beta_1$ value based on a running average from previous sessions, thus
ensuring more consistency in the predicted height.
%%%%%%%%%%%%%%%%%%%%%%%%%%%%%%%%%%%%%%%%%%%%%%%%%%%%%%%%%%%%%%%%%%%%%%%%
% Plantilla TFG/TFM
% Escuela Politécnica Superior de la Universidad de Alicante
% Realizado por: Jose Manuel Requena Plens
% Contacto: info@jmrplens.com / Telegram:@jmrplens
%%%%%%%%%%%%%%%%%%%%%%%%%%%%%%%%%%%%%%%%%%%%%%%%%%%%%%%%%%%%%%%%%%%%%%%%

\chapter{Conclusion}\label{chap:conclusion}

This work has paved the way for an innovative approach to predicting changes in
body shape during weight loss treatment, using machine learning models and the
data gathered from the previous study. Our prediction model based on \gls{smpl}
and a \gls{lstm}-based neural network successfully demonstrated its ability to
approximate the expected changes in body shape before the conclusion of the
weight loss treatment, which could potentially lead to improved adherence to
treatment plans.

\section{Objective evaluation}

In this section we will evaluate the extent to which we have achieved the
objectives set in Chapter \ref{chap:introduction}.

\subsection{Comprehensive Literature Review}

\begin{itemize}
      \item Understand the current state of the art in human body representation and
            generation.

      \item Critically analyze the advantages and disadvantages of existing methodologies.
      \item Identify the most promising approach to guide our project.
\end{itemize}

We have conducted a thorough review of the literature on human body modelling,
focusing on parametric models and neural network-based approaches, and as we
mentioned in Subsection \ref{sec:sota}, we wrote and submitted a paper to the
\gls{iwann} conference. We have identified the most promising approaches for
our project and discussed their advantages and disadvantages, which in our case
consisted of the \gls{smpl} model and generative neural networks.

\subsection{Data Preparation}

\begin{itemize}
      \item Conduct a thorough exploration of the project's data and understand its
            characteristics.
      \item Implement rigorous preprocessing steps to ensure the data's quality and
            consistency.
      \item Investigate potential data augmentation techniques to enhance the robustness of
            our models.
\end{itemize}

We extensively explored and understood the nature of the data collected from
the project. This dataset, encompassing approximately 400 sessions from 80
distinct patients, consisted of a rich array of measurements such as 3D scans,
anthropometric measures, body composition, lifestyle factors, and blood-related
data. We navigated through the challenges posed by missing data, numerous
outliers, and irregular intervals of sessions per patient. A thorough
understanding of these data characteristics led us to devise an effective data
cleaning pipeline and develop a strategy for handling unevenly distributed
sessions.

We also established a rigorous preprocessing pipeline using the \gls{pandas}
library to uphold the data's quality and consistency. We used a multi-index
system for easy patient-specific data iteration and discrepancy identification.
Tailored rules were designed to address errors due to decimal separators,
out-of-range values, and conflicting measurements.

We also paid particular attention to manual outlier detection and removal.
Measures were taken to verify that measurements were within reasonable
biological limits, such as checking that the combined muscle and fat
percentages did not exceed 100\% as well as other hand-crafted rules. Moreover,
we used the \gls{smpl} for 3D body representation, enabling us to extract
meaningful shape parameters for further analysis.

Despite our exploration for supplementary datasets being hindered by privacy
issues, static body representations, and irrelevant data fields, our efforts to
utilize data augmentation techniques were unsuccessful. The complexity of the
data proved to be an obstacle in implementing these techniques, which aimed to
artificially enlarge and diversify our existing dataset, thus improving the
model's robustness.

\subsection{Neural Network Development}

\begin{itemize}
      \item Survey potential neural network architectures suitable for time series
            prediction problems.
      \item Design a bespoke neural network architecture that is optimized for future human
            body shape generation.
      \item Implement and train the proposed neural network, fine-tuning its parameters for
            optimal performance.
      \item Evaluate the predictive performance of our trained network.
\end{itemize}

We began the process of neural network development by conducting a
comprehensive survey of potential neural network architectures suitable for
time series prediction problems. This analysis covered a wide range of
architectures including, but not limited to, \glspl{rnn}, \glspl{lstm},
\glspl{gru}, and Transformer-based models.

Through our investigation, we gained insights into the strengths and weaknesses
of each architecture in relation to our specific use case of human body shape
prediction. For example, \glspl{rnn} can suffer from the vanishing gradient
problem, making them less effective for long sequences, while \glspl{lstm} and
\glspl{gru} can handle longer sequences due to their gating mechanisms. On the
other hand, Transformers, despite their effectiveness on many tasks, can be
resource-intensive, and their applicability to time series data is still a
topic of active research.

Following the survey, we designed a bespoke neural network architecture that is
optimized for future human body shape generation. Our architecture, detailed in
Chapter \ref{chap:nn}, combines the advantages of residual and LSTM layers,
enabling it to effectively capture spatial dependencies within and temporal
dependencies across the shape descriptors.

We also incorporated batch normalization layers to stabilize the learning
process and dropout layers to mitigate overfitting. The model was designed to
be flexible and efficient, allowing it to handle a large amount of data and to
be trained within a reasonable amount of time.

Our proposed neural network was implemented using the PyTorch framework, which
provided the flexibility and efficiency necessary for our needs. The model was
trained using the \gls{adamw} optimizer and the \gls{mse} loss function, which
was deemed suitable for our regression task.

We employed a grid search to fine-tune the hyperparameters of our model.
Regular checks on the validation set were implemented to monitor the model's
performance throughout the training process. We also used early stopping to
prevent overfitting, ceasing the training process if the validation loss did
not improve over 50 consecutive epochs.

Upon training completion, we evaluated our model's predictive performance. As
can be seen in the results section, the loss function decreases rapidly in the
initial epochs, indicating the model's quick adaptation to the task. The
model's performance remained stable thereafter, with the training loss
continually decreasing, while the validation loss, though slower, did not
increase. This suggests that the model was not overfitting, a testament to the
success of our architecture design and training strategy.

\subsection{Result Evaluation}

\begin{itemize}
      \item Generate human body shape predictions using our past data.
      \item Conduct an evaluation of our model's predictive performance, discussing its
            potential applications and limitations.
\end{itemize}

Using our trained model, we generated predictions for future human body shapes
based on past data. Each prediction constituted a series of body shape
parameters representing the expected body shape of the patient after a certain
period of weight loss treatment.

Our predictive model exhibited robust performance during the validation phase,
displaying significant capability to capture and anticipate changes in body
shape parameters over time. This achievement underscores the viability of our
model for the proposed task of predicting future body shape changes in the
context of weight loss treatment.

Nevertheless, the model's predictions, while promising, were not entirely
flawless. Some inconsistencies and inaccuracies were noticed in a few cases,
which could be attributed to the inherent complexity of human body shape
changes and individual variability in weight loss patterns. These limitations
underline the need for ongoing refinement of our predictive model and its
potential extensions.

Despite these limitations, our model has substantial potential for real-world
application. If integrated into weight loss treatment programs, our model could
provide patients with a visualization of their future body shape changes,
potentially improving motivation and adherence to treatment plans.

\section{Future work}

There are many ways we could extend this work. The model's performance was
constrained by the quantity and quality of data available. Our dataset
consisted of approximately 200 sessions from 80 patients, which, while
substantial, might not fully capture the wide range of variability in human
body shapes and weight loss patterns. Moreover, outliers and missing data,
which had to be cleaned from our dataset, posed additional challenges.

Some lines of future work proposed by the author include:

\begin{itemize}
      \item \textbf{Data collection}: Patients could submit images instead of
            requiring 3D scans. This approach would allow for increased data collection,
            while also reducing friction for the patient. We can also consider
            developing a method to generate 3D models from these images.

      \item \textbf{Neural network architecture}: While we opted for a \gls{lstm}
            architecture due to the nature of our data, other neural network architectures,
            such as Transformers, could be explored for potential improvements in
            prediction performance.

      \item \textbf{Parametric models}: Other parametric models like STAR could
            be considered as alternatives or complements to the \gls{smpl}
            model we used. They might offer different advantages or better fit
            the data depending on the specific conditions and requirements.

      \item \textbf{Rendering output}: An exploration of SMPLpix \citep{prokudin2021smplpix} for 2D
            rendering output might provide an alternative approach for generating
            visualization of predicted body changes. This work uses \gls{nerf} to render
            realistic humans into \gls{smpl} models. By using the same model and the predicted
            shape parameters, we could generate realistic images of the predicted bodies instead
            of relying on 3D models for visualization.
\end{itemize}

%%%%
% CONTENIDO. BIBLIOGRAFÍA.
%%%%
\nocite{*} %incluye TODOS los documentos de la base de datos bibliográfica sean o no citados en el texto
\bibliography{bibliography/bibliography} % Archivo que contiene la bibliografía
\bibliographystyle{apacite}

%%%%
% CONTENIDO. LISTA DE ACRÓNIMOS. Comenta las líneas si no lo deseas incluir.
%%%%
% Incluye el listado de acrónimos utilizados en el trabajo. 
\printglossary[style=modsuper,type=\acronymtype,title={List of Acronyms}]
% Añade el resto de acrónimos si así se desea. Si no elimina el comando siguiente
\glsaddallunused

%%%%
% CONTENIDO. Anexos - Añade o elimina según tus necesidades
%%%%
\appendix % Inicio de los apéndices
%%%%%%%%%%%%%%%%%%%%%%%%%%%%%%%%%%%%%%%%%%%%%%%%%%%%%%%%%%%%%%%%%%%%%%%%
% Plantilla TFG/TFM
% Escuela Politécnica Superior de la Universidad de Alicante
% Realizado por: Jose Manuel Requena Plens
% Contacto: info@jmrplens.com / Telegram:@jmrplens
%%%%%%%%%%%%%%%%%%%%%%%%%%%%%%%%%%%%%%%%%%%%%%%%%%%%%%%%%%%%%%%%%%%%%%%%

\chapter{Annex I}\label{chap:annex1}
% Aquí vendría el anexo I 

\def\betaVar{3}
\def\imgWidth{0.3\textwidth}
\def\betaWidth{\textwidth}

\begin{figure}[ht!]
    \centering
    \begin{minipage}[b]{\textwidth}
        \centering
        \includegraphics[width=\imgWidth]{files/visualize_betas/beta_0_-\betaVar_m}
        \includegraphics[width=\imgWidth]{files/visualize_betas/baseline_m}
        \includegraphics[width=\imgWidth]{files/visualize_betas/beta_0_\betaVar_m}
        \linebreak
        \includegraphics[width=\imgWidth]{files/visualize_betas/beta_0_-\betaVar_f}
        \includegraphics[width=\imgWidth]{files/visualize_betas/baseline_f}
        \includegraphics[width=\imgWidth]{files/visualize_betas/beta_0_\betaVar_f}
        \caption{$\beta_1 = [-\betaVar, 0, +\betaVar]$}
        \label{fig:beta-1-vis}
    \end{minipage}
\end{figure}

\begin{figure}[ht!]
    \centering

    \begin{minipage}[b]{\textwidth}
        \centering
        \includegraphics[width=\imgWidth]{files/visualize_betas/beta_1_-\betaVar_m}
        \includegraphics[width=\imgWidth]{files/visualize_betas/baseline_m}
        \includegraphics[width=\imgWidth]{files/visualize_betas/beta_1_\betaVar_m}
        \linebreak
        \includegraphics[width=\imgWidth]{files/visualize_betas/beta_1_-\betaVar_f}
        \includegraphics[width=\imgWidth]{files/visualize_betas/baseline_f}
        \includegraphics[width=\imgWidth]{files/visualize_betas/beta_1_\betaVar_f}
        \caption{$\beta_2 = [-\betaVar, 0, +\betaVar]$}
    \end{minipage}
\end{figure}

\begin{figure}[ht!]
    \centering

    \begin{minipage}[b]{\textwidth}
        \centering
        \includegraphics[width=\imgWidth]{files/visualize_betas/beta_2_-\betaVar_m}
        \includegraphics[width=\imgWidth]{files/visualize_betas/baseline_m}
        \includegraphics[width=\imgWidth]{files/visualize_betas/beta_2_\betaVar_m}
        \linebreak
        \includegraphics[width=\imgWidth]{files/visualize_betas/beta_2_-\betaVar_f}
        \includegraphics[width=\imgWidth]{files/visualize_betas/baseline_f}
        \includegraphics[width=\imgWidth]{files/visualize_betas/beta_2_\betaVar_f}
        \caption{$\beta_3 = [-\betaVar, 0, +\betaVar]$}
    \end{minipage}
\end{figure}

\begin{figure}[ht!]
    \centering

    \begin{minipage}[b]{\textwidth}
        \centering
        \includegraphics[width=\imgWidth]{files/visualize_betas/beta_3_-\betaVar_m}
        \includegraphics[width=\imgWidth]{files/visualize_betas/baseline_m}
        \includegraphics[width=\imgWidth]{files/visualize_betas/beta_3_\betaVar_m}
        \linebreak
        \includegraphics[width=\imgWidth]{files/visualize_betas/beta_3_-\betaVar_f}
        \includegraphics[width=\imgWidth]{files/visualize_betas/baseline_f}
        \includegraphics[width=\imgWidth]{files/visualize_betas/beta_3_\betaVar_f}
        \caption{$\beta_4 = [-\betaVar, 0, +\betaVar]$}
    \end{minipage}
\end{figure}

\begin{figure}[ht!]
    \centering

    \begin{minipage}[b]{\textwidth}
        \centering
        \includegraphics[width=\imgWidth]{files/visualize_betas/beta_4_-\betaVar_m}
        \includegraphics[width=\imgWidth]{files/visualize_betas/baseline_m}
        \includegraphics[width=\imgWidth]{files/visualize_betas/beta_4_\betaVar_m}
        \linebreak
        \includegraphics[width=\imgWidth]{files/visualize_betas/beta_4_-\betaVar_f}
        \includegraphics[width=\imgWidth]{files/visualize_betas/baseline_f}
        \includegraphics[width=\imgWidth]{files/visualize_betas/beta_4_\betaVar_f}
        \caption{$\beta_5 = [-\betaVar, 0, +\betaVar]$}
    \end{minipage}
\end{figure}

\begin{figure}[ht!]
    \centering

    \begin{minipage}[b]{\textwidth}
        \centering
        \includegraphics[width=\imgWidth]{files/visualize_betas/beta_5_-\betaVar_m}
        \includegraphics[width=\imgWidth]{files/visualize_betas/baseline_m}
        \includegraphics[width=\imgWidth]{files/visualize_betas/beta_5_\betaVar_m}
        \linebreak
        \includegraphics[width=\imgWidth]{files/visualize_betas/beta_5_-\betaVar_f}
        \includegraphics[width=\imgWidth]{files/visualize_betas/baseline_f}
        \includegraphics[width=\imgWidth]{files/visualize_betas/beta_5_\betaVar_f}
        \caption{$\beta_6 = [-\betaVar, 0, +\betaVar]$}
    \end{minipage}
\end{figure}

\begin{figure}[ht!]
    \centering

    \begin{minipage}[b]{\textwidth}
        \centering
        \includegraphics[width=\imgWidth]{files/visualize_betas/beta_6_-\betaVar_m}
        \includegraphics[width=\imgWidth]{files/visualize_betas/baseline_m}
        \includegraphics[width=\imgWidth]{files/visualize_betas/beta_6_\betaVar_m}
        \linebreak
        \includegraphics[width=\imgWidth]{files/visualize_betas/beta_6_-\betaVar_f}
        \includegraphics[width=\imgWidth]{files/visualize_betas/baseline_f}
        \includegraphics[width=\imgWidth]{files/visualize_betas/beta_6_\betaVar_f}
        \caption{$\beta_7 = [-\betaVar, 0, +\betaVar]$}
    \end{minipage}
\end{figure}

\begin{figure}[ht!]
    \centering

    \begin{minipage}[b]{\textwidth}
        \centering
        \includegraphics[width=\imgWidth]{files/visualize_betas/beta_7_-\betaVar_m}
        \includegraphics[width=\imgWidth]{files/visualize_betas/baseline_m}
        \includegraphics[width=\imgWidth]{files/visualize_betas/beta_7_\betaVar_m}
        \linebreak
        \includegraphics[width=\imgWidth]{files/visualize_betas/beta_7_-\betaVar_f}
        \includegraphics[width=\imgWidth]{files/visualize_betas/baseline_f}
        \includegraphics[width=\imgWidth]{files/visualize_betas/beta_7_\betaVar_f}
        \caption{$\beta_8 = [-\betaVar, 0, +\betaVar]$}
    \end{minipage}
\end{figure}

\begin{figure}[ht!]
    \centering

    \begin{minipage}[b]{\textwidth}
        \centering
        \includegraphics[width=\imgWidth]{files/visualize_betas/beta_8_-\betaVar_m}
        \includegraphics[width=\imgWidth]{files/visualize_betas/baseline_m}
        \includegraphics[width=\imgWidth]{files/visualize_betas/beta_8_\betaVar_m}
        \linebreak
        \includegraphics[width=\imgWidth]{files/visualize_betas/beta_8_-\betaVar_f}
        \includegraphics[width=\imgWidth]{files/visualize_betas/baseline_f}
        \includegraphics[width=\imgWidth]{files/visualize_betas/beta_8_\betaVar_f}
        \caption{$\beta_9 = [-\betaVar, 0, +\betaVar]$}
    \end{minipage}
\end{figure}

\begin{figure}[ht!]
    \centering

    \begin{minipage}[b]{\textwidth}
        \centering
        \includegraphics[width=\imgWidth]{files/visualize_betas/beta_9_-\betaVar_m}
        \includegraphics[width=\imgWidth]{files/visualize_betas/baseline_m}
        \includegraphics[width=\imgWidth]{files/visualize_betas/beta_9_\betaVar_m}
        \linebreak
        \includegraphics[width=\imgWidth]{files/visualize_betas/beta_9_-\betaVar_f}
        \includegraphics[width=\imgWidth]{files/visualize_betas/baseline_f}
        \includegraphics[width=\imgWidth]{files/visualize_betas/beta_9_\betaVar_f}
        \caption{$\beta_{10} = [-\betaVar, 0, +\betaVar]$}
    \end{minipage}
\end{figure}


\input{anexos/anexo_2}
\input{anexos/anexo_3}

\end{document}