\chapter{Data analysis and preprocessing}\label{data}

Our data comprises approximately 400 sessions collected from 80 patients. These
sessions were recorded at various time points and are not uniformly spaced.
Each session's data includes a 3D scan of the patient, along with several
measurements such as weight, height, body fat percentage, etc.

\begin{table}[h]
    \centering
    \begin{tabular}{c c l c}
        \toprule
        Type                               & Source                                                                                         & Measurement (unit)                        \\
        \midrule
        \multirow{3}{*}{Anthropometric}    & \multirow{3}{4cm}{flexible measuring tape}                                                     & Wrist (cm) (kg)                           \\
                                           &                                                                                                & Waist (cm)                                \\
                                           &                                                                                                & Hip (cm)                                  \\
        \midrule

        \multirow{6}{*}{Body composition}  & \multirow{6}{4cm}{Tanita\textregistered\ MC 780-P MA and Seca\textregistered\ 213 stadiometer} & Fat per limb and trunk (\%)               \\
                                           &                                                                                                & Muscle per limb and trunk (\%)            \\
                                           &                                                                                                & Total fat and muscle (\%)                 \\
                                           &                                                                                                & Visceral fat area (cm\textsuperscript{2}) \\
                                           &                                                                                                & Weight (kg)                               \\
                                           &                                                                                                & Height (m)                                \\
        \midrule

        \multirow{3}{*}{Other, Lifestyle}  & \multirow{3}{4cm}{Interview}                                                                   & Activity   (score)                        \\
                                           &                                                                                                & Gender                                    \\
                                           &                                                                                                & Age (years)                               \\

        \midrule

        \multirow{3}{*}{Blood (capillary)} & \multirow{3}{4cm}{Accutrend\textregistered\ Plus}                                              & Glucose (mg/dL)                           \\
                                           &                                                                                                & Cholesterol (mg/dL)                       \\
                                           &                                                                                                & Triglycerides (mg/dL)                     \\

        \midrule

        \multirow{2}{*}{Blood pressure}    & \multirow{2}{4cm}{Omron\textregistered\ M3}                                                    & Systolic pressure (mmHg)                  \\
                                           &

                                           & Diastolic pressure (mmHg)                                                                                                                  \\
        \bottomrule

    \end{tabular}
    \caption{Measurements collected from each session.}
\end{table}

\todo[]{${https://rua.ua.es/dspace/bitstream/10045/124160/6/Garcia-dUrso_etal_2022_IEEEAccess.pdf}$}

However, the data requires cleaning before usage. Some sessions lack certain
measurements, and there are numerous outliers within the data. After cleaning,
we utilized about 200 sessions.

\begin{figure}[h]
    \centering
    \includegraphics[width=0.5\textwidth]{files/sessions_per_patient}
    \caption{Number of sessions per patient}
    \label{fig:sessions-per-patient}
\end{figure}

\subsection{Data cleaning}

We developed a data cleaning pipeline using the \gls{pandas} library. This
pipeline fixes some errors in the data, removes outliers and sessions with
missing measurements.

In order to do this data cleaning, we indexed the dataset with a multi-index
containing the patient's id and the number of the session. This allowed us to
iterate over the data in a per-patient basis, and to easily identify patients
with problems in their data.

After plotting and examining the data, we identified that one of the most
common issues was due to the decimal separator. Many measurements were off by a
factor of 10, so we implemented custom rules to solve this issue. We also
implemented rules to locate and remove values that were out of range or that
didn't match to other measurements.

Some of them are:

\begin{itemize}
    \item Variation between measurements of different limbs. For example, the difference
          in muscle or fat percentage between the left and right arm should be small.
    \item The sum of the fat and muscle percentages should be less than 100\%.
    \item Percentage levels of fat and muscle should be less than 100\%. If they are,
          that probably means that the measurement is off by a factor of 10.
\end{itemize}

Even after fixing these issues there were still some outliers. We tested a
system that detected extreme changes between sessions, but this was too prone
to false positives. In the end, we opted to manually visualize the data and
mark these outliers for removal.

\begin{figure}[h]
    \centering
    \begin{subfigure}{\textwidth}
        \centering
        \includegraphics[width=0.8\textwidth]{files/weight_unfiltered}
        \caption{Raw weight}
    \end{subfigure}
    \begin{subfigure}{\textwidth}
        \centering
        \includegraphics[width=0.8\textwidth]{files/weight_filtered}
        \caption{Filtered weight}
    \end{subfigure}
    \caption{Weight measurements before and after filtering}
\end{figure}

\section{Body representation}

In order to represent the body shape we opted to use \gls{smpl}. \gls{smpl}
encodes the body shape and pose using a low-dimensional linear space. The body
shape is encoded using 10 shape parameters ($\beta$), and the pose is encoded
using 72 pose parameters ($\theta$). As our interest lies in body shape, we
employed only the shape parameters.

We extracted \gls{smpl} parameters --- shape ($\beta$) and pose ($\theta$) ---
from the 3D scans using a custom minimization algorithm.

\def\betaVar{3}
\def\imgWidth{0.3\textwidth}
\def\betaWidth{\textwidth}

\begin{figure}[ht!]
    \centering
    \begin{minipage}[b]{\textwidth}
        \centering
        \includegraphics[width=\imgWidth]{files/visualize_betas/beta_0_-\betaVar_m}
        \includegraphics[width=\imgWidth]{files/visualize_betas/baseline_m}
        \includegraphics[width=\imgWidth]{files/visualize_betas/beta_0_\betaVar_m}
        \linebreak
        \includegraphics[width=\imgWidth]{files/visualize_betas/beta_0_-\betaVar_f}
        \includegraphics[width=\imgWidth]{files/visualize_betas/baseline_f}
        \includegraphics[width=\imgWidth]{files/visualize_betas/beta_0_\betaVar_f}
        \caption{$\beta_1 = [-\betaVar, 0, +\betaVar]$}
        \label{fig:beta-1-vis}
    \end{minipage}
\end{figure}

\begin{figure}[ht!]
    \centering

    \begin{minipage}[b]{\textwidth}
        \centering
        \includegraphics[width=\imgWidth]{files/visualize_betas/beta_1_-\betaVar_m}
        \includegraphics[width=\imgWidth]{files/visualize_betas/baseline_m}
        \includegraphics[width=\imgWidth]{files/visualize_betas/beta_1_\betaVar_m}
        \linebreak
        \includegraphics[width=\imgWidth]{files/visualize_betas/beta_1_-\betaVar_f}
        \includegraphics[width=\imgWidth]{files/visualize_betas/baseline_f}
        \includegraphics[width=\imgWidth]{files/visualize_betas/beta_1_\betaVar_f}
        \caption{$\beta_2 = [-\betaVar, 0, +\betaVar]$}
    \end{minipage}
\end{figure}

\begin{figure}[ht!]
    \centering

    \begin{minipage}[b]{\textwidth}
        \centering
        \includegraphics[width=\imgWidth]{files/visualize_betas/beta_2_-\betaVar_m}
        \includegraphics[width=\imgWidth]{files/visualize_betas/baseline_m}
        \includegraphics[width=\imgWidth]{files/visualize_betas/beta_2_\betaVar_m}
        \linebreak
        \includegraphics[width=\imgWidth]{files/visualize_betas/beta_2_-\betaVar_f}
        \includegraphics[width=\imgWidth]{files/visualize_betas/baseline_f}
        \includegraphics[width=\imgWidth]{files/visualize_betas/beta_2_\betaVar_f}
        \caption{$\beta_3 = [-\betaVar, 0, +\betaVar]$}
    \end{minipage}
\end{figure}

\begin{figure}[ht!]
    \centering

    \begin{minipage}[b]{\textwidth}
        \centering
        \includegraphics[width=\imgWidth]{files/visualize_betas/beta_3_-\betaVar_m}
        \includegraphics[width=\imgWidth]{files/visualize_betas/baseline_m}
        \includegraphics[width=\imgWidth]{files/visualize_betas/beta_3_\betaVar_m}
        \linebreak
        \includegraphics[width=\imgWidth]{files/visualize_betas/beta_3_-\betaVar_f}
        \includegraphics[width=\imgWidth]{files/visualize_betas/baseline_f}
        \includegraphics[width=\imgWidth]{files/visualize_betas/beta_3_\betaVar_f}
        \caption{$\beta_4 = [-\betaVar, 0, +\betaVar]$}
    \end{minipage}
\end{figure}

\begin{figure}[ht!]
    \centering

    \begin{minipage}[b]{\textwidth}
        \centering
        \includegraphics[width=\imgWidth]{files/visualize_betas/beta_4_-\betaVar_m}
        \includegraphics[width=\imgWidth]{files/visualize_betas/baseline_m}
        \includegraphics[width=\imgWidth]{files/visualize_betas/beta_4_\betaVar_m}
        \linebreak
        \includegraphics[width=\imgWidth]{files/visualize_betas/beta_4_-\betaVar_f}
        \includegraphics[width=\imgWidth]{files/visualize_betas/baseline_f}
        \includegraphics[width=\imgWidth]{files/visualize_betas/beta_4_\betaVar_f}
        \caption{$\beta_5 = [-\betaVar, 0, +\betaVar]$}
    \end{minipage}
\end{figure}

\begin{figure}[ht!]
    \centering

    \begin{minipage}[b]{\textwidth}
        \centering
        \includegraphics[width=\imgWidth]{files/visualize_betas/beta_5_-\betaVar_m}
        \includegraphics[width=\imgWidth]{files/visualize_betas/baseline_m}
        \includegraphics[width=\imgWidth]{files/visualize_betas/beta_5_\betaVar_m}
        \linebreak
        \includegraphics[width=\imgWidth]{files/visualize_betas/beta_5_-\betaVar_f}
        \includegraphics[width=\imgWidth]{files/visualize_betas/baseline_f}
        \includegraphics[width=\imgWidth]{files/visualize_betas/beta_5_\betaVar_f}
        \caption{$\beta_6 = [-\betaVar, 0, +\betaVar]$}
    \end{minipage}
\end{figure}

\begin{figure}[ht!]
    \centering

    \begin{minipage}[b]{\textwidth}
        \centering
        \includegraphics[width=\imgWidth]{files/visualize_betas/beta_6_-\betaVar_m}
        \includegraphics[width=\imgWidth]{files/visualize_betas/baseline_m}
        \includegraphics[width=\imgWidth]{files/visualize_betas/beta_6_\betaVar_m}
        \linebreak
        \includegraphics[width=\imgWidth]{files/visualize_betas/beta_6_-\betaVar_f}
        \includegraphics[width=\imgWidth]{files/visualize_betas/baseline_f}
        \includegraphics[width=\imgWidth]{files/visualize_betas/beta_6_\betaVar_f}
        \caption{$\beta_7 = [-\betaVar, 0, +\betaVar]$}
    \end{minipage}
\end{figure}

\begin{figure}[ht!]
    \centering

    \begin{minipage}[b]{\textwidth}
        \centering
        \includegraphics[width=\imgWidth]{files/visualize_betas/beta_7_-\betaVar_m}
        \includegraphics[width=\imgWidth]{files/visualize_betas/baseline_m}
        \includegraphics[width=\imgWidth]{files/visualize_betas/beta_7_\betaVar_m}
        \linebreak
        \includegraphics[width=\imgWidth]{files/visualize_betas/beta_7_-\betaVar_f}
        \includegraphics[width=\imgWidth]{files/visualize_betas/baseline_f}
        \includegraphics[width=\imgWidth]{files/visualize_betas/beta_7_\betaVar_f}
        \caption{$\beta_8 = [-\betaVar, 0, +\betaVar]$}
    \end{minipage}
\end{figure}

\begin{figure}[ht!]
    \centering

    \begin{minipage}[b]{\textwidth}
        \centering
        \includegraphics[width=\imgWidth]{files/visualize_betas/beta_8_-\betaVar_m}
        \includegraphics[width=\imgWidth]{files/visualize_betas/baseline_m}
        \includegraphics[width=\imgWidth]{files/visualize_betas/beta_8_\betaVar_m}
        \linebreak
        \includegraphics[width=\imgWidth]{files/visualize_betas/beta_8_-\betaVar_f}
        \includegraphics[width=\imgWidth]{files/visualize_betas/baseline_f}
        \includegraphics[width=\imgWidth]{files/visualize_betas/beta_8_\betaVar_f}
        \caption{$\beta_9 = [-\betaVar, 0, +\betaVar]$}
    \end{minipage}
\end{figure}

\begin{figure}[ht!]
    \centering

    \begin{minipage}[b]{\textwidth}
        \centering
        \includegraphics[width=\imgWidth]{files/visualize_betas/beta_9_-\betaVar_m}
        \includegraphics[width=\imgWidth]{files/visualize_betas/baseline_m}
        \includegraphics[width=\imgWidth]{files/visualize_betas/beta_9_\betaVar_m}
        \linebreak
        \includegraphics[width=\imgWidth]{files/visualize_betas/beta_9_-\betaVar_f}
        \includegraphics[width=\imgWidth]{files/visualize_betas/baseline_f}
        \includegraphics[width=\imgWidth]{files/visualize_betas/beta_9_\betaVar_f}
        \caption{$\beta_{10} = [-\betaVar, 0, +\betaVar]$}
    \end{minipage}
\end{figure}


Figure~\ref{fig:beta-vis} shows the effect of varying the shape parameters. We
have used a scale of 3 to better show their effect, but in practice the values
are much smaller. $\beta_0$ controls the overall height of the body, while
$\beta_1$ has a large correlation with the body mass index.

\begin{table}[h]
    \centering
    \begin{tabular}{c | c c c c c c c c c c}
        \toprule
             & $\beta_1$ & $\beta_2$ & $\beta_3$ & $\beta_4$ & $\beta_5$ & $\beta_6$ & $\beta_7$ & $\beta_8$ & $\beta_9$ & $\beta_{10}$ \\
        \midrule
        mean & 0.88      & -0.73     & 0.34      & 0.01      & 0.06      & 0.06      & 0.11      & 0.02      & 0.01      & 0.11         \\

        std  & 0.97      & 0.78      & 0.26      & 0.21      & 0.12      & 0.13      & 0.08      & 0.03      & 0.03      & 0.08         \\

        min  & -1.42     & -2.57     & -0.64     & -0.65     & -0.23     & -0.25     & -0.14     & -0.07     & -0.08     &
        -0.17                                                                                                                           \\

        25\% & 0.12      & -1.26     & 0.15      & -0.14     & -0.01     & -0.02     & 0.04      & 0.00      & -0.01     & 0.06         \\

        50\% & 0.94      & -0.69     & 0.36      & 0.03      & 0.04      & 0.02      & 0.11      & 0.02      & 0.01      & 0.11         \\

        75\% & 1.66      & -0.20     & 0.52      & 0.17      & 0.14      & 0.14      & 0.16      & 0.04      & 0.04      & 0.17         \\

        max  & 2.90      & 2.96      & 0.97      & 0.45      & 0.39      & 0.47      & 0.36      & 0.13      & 0.16      & 0.30         \\
        \bottomrule
    \end{tabular}
    \caption{Statistics of the shape parameters}
\end{table}