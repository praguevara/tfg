%%%%%%%%%%%%%%%%%%%%%%%%%%%%%%%%%%%%%%%%%%%%%%%%%%%%%%%%%%%%%%%%%%%%%%%%
% Plantilla TFG/TFM
% Escuela Politécnica Superior de la Universidad de Alicante
% Realizado por: Jose Manuel Requena Plens
% Contacto: info@jmrplens.com / Telegram:@jmrplens
%%%%%%%%%%%%%%%%%%%%%%%%%%%%%%%%%%%%%%%%%%%%%%%%%%%%%%%%%%%%%%%%%%%%%%%%

\chapter{Conclusion}\label{conclusion}

This work has paved the way for an innovative approach to predicting changes in
body shape during weight loss treatment, using machine learning models and the
data gathered from the previous study. Our prediction model based on \gls{smpl}
and a \gls{lstm}-based neural network successfully demonstrated its ability to
approximate the expected changes in body shape before the conclusion of the
weight loss treatment, which could potentially lead to improved adherence to
treatment plans.

However, our study was not without limitations. The model's performance was
constrained by the quantity and quality of data available. Our dataset
consisted of approximately 200 sessions from 80 patients, which, while
substantial, might not fully capture the wide range of variability in human
body shapes and weight loss patterns. Moreover, outliers and missing data,
which had to be cleaned from our dataset, posed additional challenges.

In future work, several improvements and expansions can be explored:

\begin{itemize}
    \item \textbf{Data collection}: Patients could submit images instead of requiring 3D scans. This approach would allow for increased data collection, while also reducing friction for the patient. We can also consider developing a method to generate 3D models from these images.
    \item \textbf{Neural network architecture}: While we opted for a \gls{lstm}
          architecture due to the nature of our data, other neural network architectures,
          such as Transformers, could be explored for potential improvements in
          prediction performance.
    \item \textbf{Parametric models}: Other parametric models like STAR could
          be considered as alternatives or complements to the \gls{smpl}
          model we used. They might offer different advantages or better fit
          the data depending on the specific conditions and requirements.

    \item \textbf{Rendering output}: An exploration of smplpix for 2D
          rendering output might provide an alternative approach for generating
          visualization of predicted body changes.
\end{itemize}