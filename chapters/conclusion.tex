%%%%%%%%%%%%%%%%%%%%%%%%%%%%%%%%%%%%%%%%%%%%%%%%%%%%%%%%%%%%%%%%%%%%%%%%
% Plantilla TFG/TFM
% Escuela Politécnica Superior de la Universidad de Alicante
% Realizado por: Jose Manuel Requena Plens
% Contacto: info@jmrplens.com / Telegram:@jmrplens
%%%%%%%%%%%%%%%%%%%%%%%%%%%%%%%%%%%%%%%%%%%%%%%%%%%%%%%%%%%%%%%%%%%%%%%%

\chapter{Conclusion}\label{chap:conclusion}

This work has paved the way for an innovative approach to predicting changes in
body shape during weight loss treatment, using machine learning models and the
data gathered from the previous study. Our prediction model based on \gls{smpl}
and a \gls{lstm}-based neural network successfully demonstrated its ability to
approximate the expected changes in body shape before the conclusion of the
weight loss treatment, which could potentially lead to improved adherence to
treatment plans.

\section{Objective evaluation}

In this section we will evaluate the extent to which we have achieved the
objectives set in Chapter \ref{chap:introduction}.

\subsection{Comprehensive Literature Review}

\begin{itemize}
      \item Understand the current state of the art in human body representation and
            generation.

      \item Critically analyze the advantages and disadvantages of existing methodologies.
      \item Identify the most promising approach to guide our project.
\end{itemize}

We have conducted a thorough review of the literature on human body modelling,
focusing on parametric models and neural network-based approaches, and as we
mentioned in Subsection \ref{sec:sota}, we wrote and submitted a paper to the
\gls{iwann} conference. We have identified the most promising approaches for
our project and discussed their advantages and disadvantages, which in our case
consisted of the \gls{smpl} model and generative neural networks.

\subsection{Data Preparation}

\begin{itemize}
      \item Conduct a thorough exploration of the project's data and understand its
            characteristics.
      \item Implement rigorous preprocessing steps to ensure the data's quality and
            consistency.
      \item Investigate potential data augmentation techniques to enhance the robustness of
            our models.
\end{itemize}

We extensively explored and understood the nature of the data collected from
the project. This dataset, encompassing approximately 400 sessions from 80
distinct patients, consisted of a rich array of measurements such as 3D scans,
anthropometric measures, body composition, lifestyle factors, and blood-related
data. We navigated through the challenges posed by missing data, numerous
outliers, and irregular intervals of sessions per patient. A thorough
understanding of these data characteristics led us to devise an effective data
cleaning pipeline and develop a strategy for handling unevenly distributed
sessions.

We also established a rigorous preprocessing pipeline using the \gls{pandas}
library to uphold the data's quality and consistency. We used a multi-index
system for easy patient-specific data iteration and discrepancy identification.
Tailored rules were designed to address errors due to decimal separators,
out-of-range values, and conflicting measurements.

We also paid particular attention to manual outlier detection and removal.
Measures were taken to verify that measurements were within reasonable
biological limits, such as checking that the combined muscle and fat
percentages did not exceed 100\% as well as other hand-crafted rules. Moreover,
we used the \gls{smpl} for 3D body representation, enabling us to extract
meaningful shape parameters for further analysis.

Despite our exploration for supplementary datasets being hindered by privacy
issues, static body representations, and irrelevant data fields, our efforts to
utilize data augmentation techniques were unsuccessful. The complexity of the
data proved to be an obstacle in implementing these techniques, which aimed to
artificially enlarge and diversify our existing dataset, thus improving the
model's robustness.

\subsection{Neural Network Development}

\begin{itemize}
      \item Survey potential neural network architectures suitable for time series
            prediction problems.
      \item Design a bespoke neural network architecture that is optimized for future human
            body shape generation.
      \item Implement and train the proposed neural network, fine-tuning its parameters for
            optimal performance.
      \item Evaluate the predictive performance of our trained network.
\end{itemize}

We began the process of neural network development by conducting a
comprehensive survey of potential neural network architectures suitable for
time series prediction problems. This analysis covered a wide range of
architectures including, but not limited to, \glspl{rnn}, \glspl{lstm},
\glspl{gru}, and Transformer-based models.

Through our investigation, we gained insights into the strengths and weaknesses
of each architecture in relation to our specific use case of human body shape
prediction. For example, \glspl{rnn} can suffer from the vanishing gradient
problem, making them less effective for long sequences, while \glspl{lstm} and
\glspl{gru} can handle longer sequences due to their gating mechanisms. On the
other hand, Transformers, despite their effectiveness on many tasks, can be
resource-intensive, and their applicability to time series data is still a
topic of active research.

Following the survey, we designed a bespoke neural network architecture that is
optimized for future human body shape generation. Our architecture, detailed in
Chapter \ref{chap:nn}, combines the advantages of residual and LSTM layers,
enabling it to effectively capture spatial dependencies within and temporal
dependencies across the shape descriptors.

We also incorporated batch normalization layers to stabilize the learning
process and dropout layers to mitigate overfitting. The model was designed to
be flexible and efficient, allowing it to handle a large amount of data and to
be trained within a reasonable amount of time.

Our proposed neural network was implemented using the PyTorch framework, which
provided the flexibility and efficiency necessary for our needs. The model was
trained using the \gls{adamw} optimizer and the \gls{mse} loss function, which
was deemed suitable for our regression task.

We employed a grid search to fine-tune the hyperparameters of our model.
Regular checks on the validation set were implemented to monitor the model's
performance throughout the training process. We also used early stopping to
prevent overfitting, ceasing the training process if the validation loss did
not improve over 50 consecutive epochs.

Upon training completion, we evaluated our model's predictive performance. As
can be seen in the results section, the loss function decreases rapidly in the
initial epochs, indicating the model's quick adaptation to the task. The
model's performance remained stable thereafter, with the training loss
continually decreasing, while the validation loss, though slower, did not
increase. This suggests that the model was not overfitting, a testament to the
success of our architecture design and training strategy.

\subsection{Result Evaluation}

\begin{itemize}
      \item Generate human body shape predictions using our past data.
      \item Conduct an evaluation of our model's predictive performance, discussing its
            potential applications and limitations.
\end{itemize}

Using our trained model, we generated predictions for future human body shapes
based on past data. Each prediction constituted a series of body shape
parameters representing the expected body shape of the patient after a certain
period of weight loss treatment.

Our predictive model exhibited robust performance during the validation phase,
displaying significant capability to capture and anticipate changes in body
shape parameters over time. This achievement underscores the viability of our
model for the proposed task of predicting future body shape changes in the
context of weight loss treatment.

Nevertheless, the model's predictions, while promising, were not entirely
flawless. Some inconsistencies and inaccuracies were noticed in a few cases,
which could be attributed to the inherent complexity of human body shape
changes and individual variability in weight loss patterns. These limitations
underline the need for ongoing refinement of our predictive model and its
potential extensions.

Despite these limitations, our model has substantial potential for real-world
application. If integrated into weight loss treatment programs, our model could
provide patients with a visualization of their future body shape changes,
potentially improving motivation and adherence to treatment plans.

\section{Future work}

There are many ways we could extend this work. The model's performance was
constrained by the quantity and quality of data available. Our dataset
consisted of approximately 200 sessions from 80 patients, which, while
substantial, might not fully capture the wide range of variability in human
body shapes and weight loss patterns. Moreover, outliers and missing data,
which had to be cleaned from our dataset, posed additional challenges.

Some lines of future work proposed by the author include:

\begin{itemize}
      \item \textbf{Data collection}: Patients could submit images instead of
            requiring 3D scans. This approach would allow for increased data collection,
            while also reducing friction for the patient. We can also consider
            developing a method to generate 3D models from these images.

      \item \textbf{Neural network architecture}: While we opted for a \gls{lstm}
            architecture due to the nature of our data, other neural network architectures,
            such as Transformers, could be explored for potential improvements in
            prediction performance.

      \item \textbf{Parametric models}: Other parametric models like STAR could
            be considered as alternatives or complements to the \gls{smpl}
            model we used. They might offer different advantages or better fit
            the data depending on the specific conditions and requirements.

      \item \textbf{Rendering output}: An exploration of SMPLpix \citep{prokudin2021smplpix} for 2D
            rendering output might provide an alternative approach for generating
            visualization of predicted body changes.
\end{itemize}